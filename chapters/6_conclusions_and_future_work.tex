%!TEX root = ../projecto.tex

\section{Conclusions and Future Work} % (fold)
\label{sec:coclusions_future_work}

In this report we presented a new approach to topic detection in datasets with a short documents, by leveraging the implicit semantics behind the data. 
First, in Section~\ref{sub:introduction}, we started by introducing the concepts and problems of topic detection the Twitter social network. Afterwards at Section~\ref{sec:basic_concepts}, we described basic concepts for this project, such as how document clustering and Self-Organizing Maps work.
Then at Section~\ref{sec:related_work} we focused on describing related work done by other researchers in the field of Self-Organizing Maps, topic detection with clustering and data mining on Twitter. At Section~\ref{sec:architecture} we proposed how we where going to use and adapt the SOM algorithm to better fit for solving the problem of topic detection on Twitter. In this section we also described the architecture of our prototype which will used to demonstrate live topic categorization. Finally at Section~\ref{sec:evaluation_metrics} we described how we where going to evaluate our solution.
\\
Even though this project is a prof of concept on topic detection by focusing on user specific interests and relations, and on the application of Self-Organizing Maps as a tool for clustering user tweets with social features. We think that by escalating the proposed architecture in order for it to be able to process a lot more data from Twitter will render interesting results. We presume this by looking at work done in other research areas like work from ~\citet{Le2011} that also used neural networks for unsupervised learning but leveraged giant datasets and computational power. 

% section coclusions_future_work (end)
