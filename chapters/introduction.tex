\section{Introduction}

Why use topic detection
Why use clustering
Why use SOMs

\textit{ \textbf{Setting the context:}} 
Twitter is the most popular microblogging platform that enables users to share their own thoughts in less than 140 characters. Through out the years twitter evolved from a simple platform to share what a person is doing into an message broadcaster. Nowadays twitter is used in every kind of areas, from journalism to political campaigns. 
The way users engage in the social network was enhanced by adding Hashtags a way to tag a certain topic. Sharing web content through urls in order to direct link a certain article or blog post.Responses can be made to a user twit in order to engage users to talk directly about a certain issue and retweets a way to broadcast a tweet from another user.

\textit{ \textbf{clearly communicate what you want to discover:}} 
In this report we will introduce the concept of using Self Organizing Maps in order to cluster tweets based on the Social Network. What we hope to acheive is a new way to find topic detection based on the concept of Homophily which states that people tend to associate and bond with similar others. If people tend to follow other people with the same interests it is possible that groups of people tend to tweet about certain topics.

In the end of the project, it will be possible to use a web interface to see topic-clusters based on the Twitter social interactions.

\subsection{Self Organizing Maps Usage} % (fold)
\label{sub:self_organizing_maps_usage}


% subsection self_organizing_maps_usage (end)

\begin{itemize}
  \settowidth{\leftmargin}{{\Large$\square$}}\advance\leftmargin\labelsep
  
  \renewcommand\labelitemi{{\lower1.5pt\hbox{\Large$\square$}}}
  \item Set the context; 
  \item explain the situation;
  \item State why the main idea is important;  
  \item provide general information about the main idea
  \item clearly communicate what you want to discover why you are interested in the topic
  \item Outline the structure 
\end{itemize}
