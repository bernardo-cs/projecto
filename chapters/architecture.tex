\section{Architecture}

In this project we are going to apply Self Organizing Maps in order to detect clusters of Topics on Twitter. 

\subsection{Data Gathering} % (fold)
 \label{sub:data_gathering}

 In order to retrieve data from Twitter, we will be using a ruby library called Twitter Stream \cite{TweetStream}, that enables the user to download and inspect the twitter stream of tweets. As the data is gathered it will be stored in a MongoDB database for posterior analyses. 
 As the twitter stream is stored, another function will interact with the twitter API in order to retrieve information from a user profile and relate him with other users by analyzing his followers and who the user is following.
 In the end of the data-gathering process it will be possible to query the database for:
 \begin{itemize}
   \item  Tweets from a user.
   \item  Query tweets for hashtag.
   \item  Query users followers and who he is following.
   \item  Query for tweets that shared the same URI 
 \end{itemize}

 % subsection data_gathering (end) 

\subsection{Data Characterization} % (fold)
\label{sub:data_labeling}
Depois de se tirar twits a partir da API de streaming do mesmo, vamos contar as palavras mais utilizadas na rede social depois de se remover stop words, palavrões e abreviaturas socias (como lol, omg, brb e combinações das mesmas). As palavras com maior ocurrencia vão ser utilizadas como identificadores num tweet. De seguida para cada utilizador conta-se a quantidade de vezes que cada um mencionou cada uma das palavras com maior frequencia no twitter, desta forma identificamos o tipo de conteudo que um utilizador tem tendecia a produzir. 
De forma a criarmos a representação de rede social iremos passar 30 por cento das palavras mais escritas de um utilizador para os seus followers e acrescentalas ao raking do mesmo.

Os SOM vão organizar os teweets com base na representação social de um utilizador, deste modo esperamos encontrar núcleos de tweets com utilizadores com formas de escrita e interesses parecidos que faça com  que os seus tweets sejam sobre tópico similares.

De forma a se fazer topic detection iremos pegar em palavras chave de um determinado tópico e porcurar em clusters que apresentem o maior numero de utlização dessas palavras.


% subsection data_labeling (end)

\begin{itemize}
  \item How am I gonna solve the problem? 
  \item Describe the work that will be done 
\end{itemize}
