\section{Related Work}

\begin{itemize}
  \item What did we know about the problem before I did this study? 
  \item What did we do different from previous works? 
  \item Discuss the relevant primary research literature 
  \item Works should be organized by their relevant characteristics 
  \item Comment on why it is relevant for your work 
  \item Comment on what your work does differently 
\end{itemize}

\subsection{Clustering and Self Organizing Maps} % (fold)
\label{sub:self_organizing_maps}

Cheong at al\cite{Cheong2010} analyzed user profile characteristics based on a certain topic using SOM. On three different topics, "IPhone Software Launch", "Obamas Foreign Policy" an "Iran Election" they could find distinct types of users twitting:

\begin{itemize}
  \item "IPhone Software Launch"
  \begin{itemize}
    \item Major twits where from male users, with accounts greater than 90 days, coming from countries where iPhone was marketed, with high adoption of blogs or social media sites.
    \item Twits with higher ratio of followers to followees with high frequency of twitter posts per day, twits with links to shared content, no country or gender specificity. Typically a news aggregator or news organization.
    \item One day twit account, with unpopular social connections, lacking profile customization, frequently posts more than 50 tweets daily with URIs. Characterizing a typical spammer.
  \end{itemize}
\end{itemize}

\begin{itemize}
  \item "Iran Election"
  \begin{itemize}
    \item Recently registered, Iranian web-based twitter users, frequent patterns of replying
    \item Users from every where in the world, long message sizes. Users trying to raise awareness
    \item Users with accounts older than 3 months, contribute sparingly to Twitter, but have a high usage of other social media sites.
    \item Variance in Twitter account and nationality, who frequently posted URL links in messages
  \end{itemize}
\end{itemize}

\begin{itemize}
  \item Obama’s foreign policy
  \begin{itemize}
    \item American residents discussing the topic, accounts more than three months old, their messages are almost always long, and their messaging style is focused towards replies. Describing users talking about the issue
    \item Users with many followers, predominantly US males, URI links in their messages. Which describes news sources and opinion leaders.
    \item Mainly new accounts from everywhere around the world which arises suspicions of marketing/opinion spam.
  \end{itemize}
\end{itemize}

It is possible to see based on their results, that there is direct relation between a twitter profile characteristics and the content produced by the user. Even though in the paper is not described the amount of signals used, it is possible to determine the following:
\begin{itemize}
  \item The time when an account was created.
  \item If a tweets is used to start a thread.
  \item The size of a tweet.
  \item Localization of the tweet.
  \item Number of followers and followees.
  \item Number of contributions to twitter.
  \item Profile customization.
  \item Popularity of the connections (number of followers of the followers)
  \item Content of the URI´s
  \item Number of tweets a day.
\end{itemize}

% subsection self_organizing_maps (end)

\subsection{Topic Detection on Twitter} % (fold)
\label{sub:topic_detection_on_twitter}

% subsection topic_detection_on_twitter (end)

