%!TEX root = ../projecto.tex

\section{Introduction}

With the evolution of social networks websites like Facebook or Twitter throughout the last couple of years, the amount of pertinent content about a specif issue is increasing dramatically, which calls for new ways to make sense and catalog this data.
In the other hand find topic sensitive information on social networks is extremely complicated due to the fact that documents have very little content, slang vocabulary and orthographically mistakes or abbreviations.

The value of data presented in sites like Facebook or Twitter as proven its value in papers like “Predicting the future with social media” where it is possible to predict with high precision the value of a movie box office weeks before it debuts.

This project will focus on Topic Detection on Twitter by using a new approach that will not only take in consideration the words in the corpus, but will also take in account the social network to which a tweet belongs to in order to categorize it using the concept of homophily that has been proven in past that is applicable to social networks.

Will be described the objectives of this project, at \ref{sec:basic_concepts} we will talk about the state of the art solutions related not only to topic detection but also to twitter data analysis and Self-Organizing Maps.In section \ref{sec:architecture} Architecture of the purposed solution and finally at section \ref{sec:evaluation_metrics} it will be discussed how to evaluate results achieved.

\subsection{Objectives} % (fold)
\label{sub:objectives}

The objective of this project is clear, finding topics on Tweets by analyzing their corpus specific characteristics, like number of characters in a tweet, hashtag, “was retweeted”, etc.. And contextualize the social network evolving the person that did the tweet.

After characterizing the tweet with information just described, we will use the unsupervised learning clustering technique Self-organizing maps in order to organize the tweets in clusters of topics. Afterwards it will be needed to categorize the clusters in order to know which topic they belong to.

Lastly the resulting topic clusters will be publicly accessible through a website to everybody that visits it.

% section objectives (end)
% subsection objectives (end)
