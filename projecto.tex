
%%%%%%%%%%%%%%%%%%%%%%% file typeinst.tex %%%%%%%%%%%%%%%%%%%%%%%%%
%
% This is the LaTeX source for the instructions to authors using
% the LaTeX document class 'llncs.cls' for contributions to
% the Lecture Notes in Computer Sciences series.
% http://www.springer.com/lncs       Springer Heidelberg 2006/05/04
%
% It may be used as a template for your own input - copy it
% to a new file with a new name and use it as the basis
% for your article.
%
% NB: the document class 'llncs' has its own and detailed documentation, see
% ftp://ftp.springer.de/data/pubftp/pub/tex/latex/llncs/latex2e/llncsdoc.pdf
%
%%%%%%%%%%%%%%%%%%%%%%%%%%%%%%%%%%%%%%%%%%%%%%%%%%%%%%%%%%%%%%%%%%%


\documentclass[runningheads,a4paper]{llncs}

\usepackage{amssymb}
\setcounter{tocdepth}{3}
\usepackage{graphicx}
%%%%
\usepackage{float}
\usepackage[absolute]{textpos}
\usepackage{tabularx}                 
\usepackage{fancyhdr}
\usepackage{helvet} 
\usepackage{ucs}
%%%%
\usepackage{url}
\usepackage[utf8x]{inputenc}
\usepackage[english]{babel}
\usepackage[numbers]{natbib}

\urldef{\mailsa}\path|bernardo.simoes@ist.ul.pt|
\newcommand{\keywords}[1]{\par\addvspace\baselineskip
\noindent\keywordname\enspace\ignorespaces#1}
\bibliographystyle{plainnat}

\begin{document}

\mainmatter  % start of an individual contribution

% first the title is needed
\title{SocialSOM: \\Topic Detection on Twitter by Organizing Tweets on User Similarity}

% a short form should be given in case it is too long for the running head
\titlerunning{SocialSOM}

% the name(s) of the author(s) follow(s) next
%
% NB: Chinese authors should write their first names(s) in front of
% their surnames. This ensures that the names appear correctly in
% the running heads and the author index.
%
\author{Bernardo Simões  20-25 páginas aproximadamente}
%
\authorrunning{Bernardo Simões}
% (feature abused for this document to repeat the title also on left hand pages)

% the affiliations are given next; don't give your e-mail address
% unless you accept that it will be published
\institute{Technical University of Lisbon - Taguspark Campus,\\
Av. Prof. Doutor An\'{\i}bal Cavaco Silva — 2744-016 Porto Salvo, Portugal\\
\mailsa\\
\url{http://www.ist.utl.pt/en/}}

%
% NB: a more complex sample for affiliations and the mapping to the
% corresponding authors can be found in the file "llncs.dem"
% (search for the string "\mainmatter" where a contribution starts).
% "llncs.dem" accompanies the document class "llncs.cls".
%

\toctitle{Lecture Notes in Computer Science}
\tocauthor{Authors' Instructions}
\maketitle


\begin{abstract}

\emph{70 and at most 150 words, What did I do, in a nutshell?, summarize the paper, should be written last , very short context ,what the objectives of the study were }
\keywords{topic detection, twitter, self-organizing maps, classification, clustering}
\end{abstract}

%!TEX root = ../projecto.tex

\section{Introduction}

With the evolution of social networks websites like Facebook or Twitter throughout the last couple of years, the amount of pertinent content about a specif issue is increasing dramatically, which calls for new ways to make sense and catalog this data.
In the other hand find topic sensitive information on social networks is extremely complicated due to the fact that documents have very little content, slang vocabulary and orthographically mistakes or abbreviations.

The value of data presented in sites like Facebook or Twitter as proven its value in papers like “Predicting the future with social media” where it is possible to predict with high precision the value of a movie box office weeks before it debuts.

This project will focus on Topic Detection on Twitter by using a new approach that will not only take in consideration the words in the corpus, but will also take in account the social network to which a tweet belongs to in order to categorize it using the concept of homophily that has been proven in past that is applicable to social networks.

Will be described the objectives of this project, at \ref{sec:related_work} we will talk about the state of the art solutions related not only to topic detection but also to twitter data analysis and Self-Organizing Maps.In section \ref{sec:architecture} Architecture of the purposed solution and finally at section \ref{sec:evaluation_metrics} it will be discussed how to evaluate results achieved.

\subsection{Objectives} % (fold)
\label{sub:objectives}

The objective of this project is clear, finding topics on Tweets by analyzing their corpus specific characteristics, like number of characters in a tweet, hashtag, “was retweeted”, etc.. And contextualize the social network evolving the person that did the tweet.

After characterizing the tweet with information just described, we will use the unsupervised learning clustering technique Self-organizing maps in order to organize the tweets in clusters of topics. Afterwards it will be needed to categorize the clusters in order to know which topic they belong to.

Lastly the resulting topic clusters will be publicly accessible through a website to everybody that visits it.

% section objectives (end)
% subsection objectives (end)

%!TEX root = ../projecto.tex
\section{Related Work} % (fold)
\label{sec:related_work}

% section related_work (end)
\begin{itemize}
  \item What did we know about the problem before I did this study? 
  \item What did we do different from previous works? 
  \item Discuss the relevant primary research literature 
  \item Works should be organized by their relevant characteristics 
  \item Comment on why it is relevant for your work 
  \item Comment on what your work does differentely 
\end{itemize}

\subsection{Clustering and Self Organizing Maps} % (fold)
\label{sub:self_organizing_maps}

% subsection self_organizing_maps (end)

\subsection{Topic Detection on Twitter} % (fold)
\label{sub:topic_detection_on_twitter}
% subsection topic_detection_on_twitter (end)
%!TEX root = ../projecto.tex

\section{Architecture} % (fold)
\label{sec:architecture}
In this project we are going to apply Self Organizing Maps in order to detect clusters of Topics on Twitter. 
% section architecture (end)

\subsection{Data Gathering} % (fold)
 \label{sub:data_gathering}

 In order to retrieve data from Twitter, we will be using a ruby library called Twitter Stream \cite{TweetStream} , that enables the user to download and inspect the twitter stream of tweets. As the data is gathered it will be stored in a MongoDB database for posterior analyses. 
 As the twitter stream is stored, another function will interact with the twitter API in order to retrieve information from a user profile and relate him with other users by analyzing his followers and who the user is following.
 In the end of the data-gathering process it will be possible to query the database for:
 \begin{itemize}
   \item  Tweets from a user.
   \item  Query tweets for hashtag.
   \item  Query users followers and who he is following.
   \item  Query for tweets that shared the same URI 
 \end{itemize}

 Tweets will be cathegorized in:
 \begin{itemize}
   \item News Accounts
   \begin{itemize}
     \item Accounts with a lot of followers
   \end{itemize}
   \item Profile custumization 
   \item Average number of tweets a day with uri (might suggest spam)
 \end{itemize}

 % subsection data_gathering (end) 

\begin{itemize}
  \item How am I gonna solve the problem? 
  \item Describe the work that will be done 
\end{itemize}

%!TEX root = ../projecto.tex

\section{Evaluation Metrics} % (fold)
\label{sec:evaluation_metrics}

% section evaluation_metrics (end)

\begin{itemize}
  \item How am I gonna evaluate my work?
\end{itemize}

\subsection{Evaluation Criteria by Teachers} % (fold)
\label{sub:evaluation_criteria}
\begin{itemize}
  \item Ability to understand the research problem
  \item Clear and well defined goals
  \item Description of the different approaches explored
  \item Ability to relate the state-of-the-art with the research theme Work methodology and adequate planning for the next stage Organization and quality of the written document
  \item Inclusion and completeness of updated and appropriate references Oral presentation and discussion
\end{itemize}
% subsection evaluation_criteria (end)

\bibliographystyle{plain}
\bibliography{/Users/bersimoes/Documents/Tese/PaperProjecto/Bibtex/library,/Users/bersimoes/Documents/Tese/PaperProjecto/Bibtex/Sites}

\end{document}
