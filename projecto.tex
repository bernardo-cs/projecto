
%%%%%%%%%%%%%%%%%%%%%%% file typeinst.tex %%%%%%%%%%%%%%%%%%%%%%%%%
%
% This is the LaTeX source for the instructions to authors using
% the LaTeX document class 'llncs.cls' for contributions to
% the Lecture Notes in Computer Sciences series.
% http://www.springer.com/lncs       Springer Heidelberg 2006/05/04
%
% It may be used as a template for your own input - copy it
% to a new file with a new name and use it as the basis
% for your article.
%
% NB: the document class 'llncs' has its own and detailed documentation, see
% ftp://ftp.springer.de/data/pubftp/pub/tex/latex/llncs/latex2e/llncsdoc.pdf
%
%%%%%%%%%%%%%%%%%%%%%%%%%%%%%%%%%%%%%%%%%%%%%%%%%%%%%%%%%%%%%%%%%%%


\documentclass[runningheads,a4paper]{llncs}

\usepackage{amssymb}
\setcounter{tocdepth}{3}
\usepackage{graphicx}

\usepackage{url}
\usepackage[utf8x]{inputenc}
\urldef{\mailsa}\path|bernardo.simoes@ist.ul.pt|
\newcommand{\keywords}[1]{\par\addvspace\baselineskip
\noindent\keywordname\enspace\ignorespaces#1}

\begin{document}

\mainmatter  % start of an individual contribution

% first the title is needed
\title{SocialSOM: \\Topic Detection on Twitter by Organizing Tweets on User Similarity}

% a short form should be given in case it is too long for the running head
\titlerunning{SocialSOM}

% the name(s) of the author(s) follow(s) next
%
% NB: Chinese authors should write their first names(s) in front of
% their surnames. This ensures that the names appear correctly in
% the running heads and the author index.
%
\author{Bernardo Simões  20-25 páginas aproximadamente}
%
\authorrunning{Bernardo Simões}
% (feature abused for this document to repeat the title also on left hand pages)

% the affiliations are given next; don't give your e-mail address
% unless you accept that it will be published
\institute{Technical University of Lisbon - Taguspark Campus,\\
Av. Prof. Doutor An\'{\i}bal Cavaco Silva — 2744-016 Porto Salvo, Portugal\\
\mailsa\\
\url{http://www.ist.utl.pt/en/}}

%
% NB: a more complex sample for affiliations and the mapping to the
% corresponding authors can be found in the file "llncs.dem"
% (search for the string "\mainmatter" where a contribution starts).
% "llncs.dem" accompanies the document class "llncs.cls".
%

\toctitle{Lecture Notes in Computer Science}
\tocauthor{Authors' Instructions}
\maketitle


\begin{abstract}

\emph{70 and at most 150 words, What did I do, in a nutshell?, summarize the paper, should be written last , very short context ,what the objectives of the study were }
\keywords{topic detection, twitter, self-organizing maps, classification, clustering}
\end{abstract}

\section{Introduction}

Why use topic detection
Why use clustering
Why use SOMs

\textit{ \textbf{Setting the context:}} 
Twitter is the most popular microblogging platform that enables users to share their own thoughts in less than 140 characters. Through out the years twitter evolved from a simple platform to share what a person is doing into an message broadcaster. Nowadays twitter is used in every kind of areas, from journalism to political campaigns. 
The way users engage in the social network was enhanced by adding Hashtags a way to tag a certain topic. Sharing web content through urls in order to direct link a certain article or blog post.Responses can be made to a user twit in order to engage users to talk directly about a certain issue and retweets a way to broadcast a tweet from another user.

\textit{ \textbf{clearly communicate what you want to discover:}} 
In this report we will introduce the concept of using Self Organizing Maps in order to cluster tweets based on the Social Network. What we hope to acheive is a new way to find topic detection based on the concept of Homophily which states that people tend to associate and bond with similar others. If people tend to follow other people with the same interests it is possible that groups of people tend to tweet about certain topics.

In the end of the project, it will be possible to use a web interface to see topic-clusters based on the Twitter social interactions.

\subsection{Self Organizing Maps Usage} % (fold)
\label{sub:self_organizing_maps_usage}


% subsection self_organizing_maps_usage (end)

\begin{itemize}
  \settowidth{\leftmargin}{{\Large$\square$}}\advance\leftmargin\labelsep
  
  \renewcommand\labelitemi{{\lower1.5pt\hbox{\Large$\square$}}}
  \item Set the context; 
  \item explain the situation;
  \item State why the main idea is important;  
  \item provide general information about the main idea
  \item clearly communicate what you want to discover why you are interested in the topic
  \item Outline the structure 
\end{itemize}

\section{Objectives}
\section{Related Work}

\begin{itemize}
  \item What did we know about the problem before I did this study? 
  \item What did we do different from previous works? 
  \item Discuss the relevant primary research literature 
  \item Works should be organized by their relevant characteristics 
  \item Comment on why it is relevant for your work 
  \item Comment on what your work does differently 
\end{itemize}

\subsection{Clustering and Self Organizing Maps} % (fold)
\label{sub:self_organizing_maps}

Cheong at al\cite{Cheong2010} analyzed user profile characteristics based on a certain topic using SOM. On three different topics, "IPhone Software Launch", "Obamas Foreign Policy" an "Iran Election" they could find distinct types of users twitting:

\begin{itemize}
  \item "IPhone Software Launch"
  \begin{itemize}
    \item Major twits where from male users, with accounts greater than 90 days, coming from countries where iPhone was marketed, with high adoption of blogs or social media sites.
    \item Twits with higher ratio of followers to followees with high frequency of twitter posts per day, twits with links to shared content, no country or gender specificity. Typically a news aggregator or news organization.
    \item One day twit account, with unpopular social connections, lacking profile customization, frequently posts more than 50 tweets daily with URIs. Characterizing a typical spammer.
  \end{itemize}
\end{itemize}

\begin{itemize}
  \item "Iran Election"
  \begin{itemize}
    \item Recently registered, Iranian web-based twitter users, frequent patterns of replying
    \item Users from every where in the world, long message sizes. Users trying to raise awareness
    \item Users with accounts older than 3 months, contribute sparingly to Twitter, but have a high usage of other social media sites.
    \item Variance in Twitter account and nationality, who frequently posted URL links in messages
  \end{itemize}
\end{itemize}

\begin{itemize}
  \item Obama’s foreign policy
  \begin{itemize}
    \item American residents discussing the topic, accounts more than three months old, their messages are almost always long, and their messaging style is focused towards replies. Describing users talking about the issue
    \item Users with many followers, predominantly US males, URI links in their messages. Which describes news sources and opinion leaders.
    \item Mainly new accounts from everywhere around the world which arises suspicions of marketing/opinion spam.
  \end{itemize}
\end{itemize}

It is possible to see based on their results, that there is direct relation between a twitter profile characteristics and the content produced by the user. Even though in the paper is not described the amount of signals used, it is possible to determine the following:
\begin{itemize}
  \item The time when an account was created.
  \item If a tweets is used to start a thread.
  \item The size of a tweet.
  \item Localization of the tweet.
  \item Number of followers and followees.
  \item Number of contributions to twitter.
  \item Profile customization.
  \item Popularity of the connections (number of followers of the followers)
  \item Content of the URI´s
  \item Number of tweets a day.
\end{itemize}

% subsection self_organizing_maps (end)

\subsection{Topic Detection on Twitter} % (fold)
\label{sub:topic_detection_on_twitter}

% subsection topic_detection_on_twitter (end)


\section{Architecture}

In this project we are going to apply Self Organizing Maps in order to detect clusters of Topics on Twitter. 

\subsection{Data Gathering} % (fold)
 \label{sub:data_gathering}

 In order to retrieve data from Twitter, we will be using a ruby library called Twitter Stream \cite{TweetStream}, that enables the user to download and inspect the twitter stream of tweets. As the data is gathered it will be stored in a MongoDB database for posterior analyses. 
 As the twitter stream is stored, another function will interact with the twitter API in order to retrieve information from a user profile and relate him with other users by analyzing his followers and who the user is following.
 In the end of the data-gathering process it will be possible to query the database for:
 \begin{itemize}
   \item  Tweets from a user.
   \item  Query tweets for hashtag.
   \item  Query users followers and who he is following.
   \item  Query for tweets that shared the same URI 
 \end{itemize}

 % subsection data_gathering (end) 

\subsection{Data Characterization} % (fold)
\label{sub:data_labeling}
Depois de se tirar twits a partir da API de streaming do mesmo, vamos contar as palavras mais utilizadas na rede social depois de se remover stop words, palavrões e abreviaturas socias (como lol, omg, brb e combinações das mesmas). As palavras com maior ocurrencia vão ser utilizadas como identificadores num tweet. De seguida para cada utilizador conta-se a quantidade de vezes que cada um mencionou cada uma das palavras com maior frequencia no twitter, desta forma identificamos o tipo de conteudo que um utilizador tem tendecia a produzir. 
De forma a criarmos a representação de rede social iremos passar 30 por cento das palavras mais escritas de um utilizador para os seus followers e acrescentalas ao raking do mesmo.

Os SOM vão organizar os teweets com base na representação social de um utilizador, deste modo esperamos encontrar núcleos de tweets com utilizadores com formas de escrita e interesses parecidos que faça com  que os seus tweets sejam sobre tópico similares.

De forma a se fazer topic detection iremos pegar em palavras chave de um determinado tópico e porcurar em clusters que apresentem o maior numero de utlização dessas palavras.


% subsection data_labeling (end)

\begin{itemize}
  \item How am I gonna solve the problem? 
  \item Describe the work that will be done 
\end{itemize}

\section{Evaluation Metrics}

\begin{itemize}
  \item How am I gonna evaluate my work?
\end{itemize}

\subsection{Evaluation Criteria by Teachers} % (fold)
\label{sub:evaluation_criteria}
\begin{itemize}
  \item Ability to understand the research problem
  \item Clear and well defined goals
  \item Description of the different approaches explored
  \item Ability to relate the state-of-the-art with the research theme Work methodology and adequate planning for the next stage Organization and quality of the written document
  \item Inclusion and completeness of updated and appropriate references Oral presentation and discussion
\end{itemize}
% subsection evaluation_criteria (end)

\bibliographystyle{plain}
\bibliography{/Users/bersimoes/Documents/Tese/PaperProjecto/Bibtex/Clustering.bib,/Users/bersimoes/Documents/Tese/PaperProjecto/Bibtex/Sites.bib,/Users/bersimoes/Documents/Tese/PaperProjecto/Bibtex/SOMs.bib}{}


\end{document}
